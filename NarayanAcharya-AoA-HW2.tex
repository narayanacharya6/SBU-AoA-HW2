\documentclass[11pt]{article}
% Packages used by instructor %
\usepackage{amsmath,amssymb,xspace,epsfig}

% Package for new section on new page
\usepackage{titlesec}

%Package used for formatting lists%
\usepackage[shortlabels]{enumitem}

% Package used for setting up page margins %
\usepackage{geometry}
%\usepackage{showframe} % Used to clearly show the new margins %
\newgeometry{vmargin={1in}, hmargin={1in,1in}}

% Package used for adding hyperlinks %
\usepackage{hyperref}

% Package for multirow tables %
\usepackage{multirow}
\renewcommand{\arraystretch}{1.5}

% Package for displaying algorithms %
\usepackage{algorithm}
\usepackage{algpseudocode}

% Package for Bib in ToC %
\usepackage[nottoc]{tocbibind}

% Break algorithm across pages %
\makeatletter
\newenvironment{breakablealgorithm}
{% \begin{breakablealgorithm}
	\begin{center}
		\refstepcounter{algorithm}% New algorithm
		\hrule height.8pt depth0pt \kern2pt% \@fs@pre for \@fs@ruled
		\renewcommand{\caption}[2][\relax]{% Make a new \caption
			{\raggedright\textbf{\ALG@name~\thealgorithm} ##2\par}%
			\ifx\relax##1\relax % #1 is \relax
			\addcontentsline{loa}{algorithm}{\protect\numberline{\thealgorithm}##2}%
			\else % #1 is not \relax
			\addcontentsline{loa}{algorithm}{\protect\numberline{\thealgorithm}##1}%
			\fi
			\kern2pt\hrule\kern2pt
		}
	}{% \end{breakablealgorithm}
		\kern2pt\hrule\relax% \@fs@post for \@fs@ruled
	\end{center}
}
\makeatother

% Page Header %
\usepackage{fancyhdr}
\pagestyle{fancy}
\fancyhf{}
\lhead{Ananlysis of Algorithms \\ Home Work 2}
\rhead{Narayan Acharya \\ 112734365}
\lfoot{\leftmark}
\rfoot{\thepage}

% Do not indent pargraphs everywhere %
\setlength\parindent{0pt}

\renewcommand{\headrulewidth}{1pt}
\renewcommand{\footrulewidth}{1pt}

\title{
	Analysis of Algorithms - Home Work 2\\[2mm]
	\large CSE 548 Fall '19\\[1mm]
	\href{mailto:jgao@cs.stonybrook.edu}{\textit{Prof. Jie Gao}}
}
\author{
	\small Submission By: \\
	\href{mailto:nacharya@cs.stonybrook.edu}{Narayan Acharya} \\
	\small 112734365
}
\date{\vspace{-5ex}}
\begin{document}

% Set up title
\maketitle
\thispagestyle{fancy} % Make this page use fancy header

% Set up Table of Contents
\tableofcontents

\clearpage

\section{Question 1} Textbook [Kleinberg \& Tardos] Chapter 3, page 107, problem \#6. \\
\textbf{Solution:} \\

We know that tree $ T $ is both a DFS tree and a BFS tree. So tree T should exhibit properties of both DFS and BFS trees. \\

For a DFS tree for two nodes \textbf{not} to be connected by an existing edge, one must be an ancestor of the other. [See Kleinberg \& Tardos, proof 3.7, page 85] \\

For a BFS tree for two nodes to be connected, their distance from another node $ w $ in $ T $ can differ by at most 1. [See Kleinberg \& Tardos, proof 3.4, page 81] \\

Suppose, there exists an edge $ e $ that connects nodes $ u $ and $ v $ in our graph $ G $ but does \textbf{NOT} belong to edges $ E $ of tree $ T $, i.e. $ e=(u,v) $ and $ e \notin E $. \\

As $ u $ and $ v $ are connected by an edge \textbf{not} in $ T $, one of them is an ancestor to the other. Without loss of generality, assume node $ u $ is an ancestor of $ v $. Given $ T $ is also BFS tree, distance from random node $ w $ in $ T $ to $ u $ and $ v $ can differ by a maximum of 1. Thus $ u $ has to be a direct parent of $ v $. This implies that edge $ e $ connecting them must be a part of $ T $. \\

This contradicts our initial assumption of $ e \notin E $. Hence $ e \in E $.
   
\clearpage
\section{Question 2} Textbook [Kleinberg \& Tardos] Chapter 3, page 107, problem \#8. \\


\clearpage
\section{Question 3} Textbook [Kleinberg \& Tardos] Chapter 3, page 107, problem \#12. \\


\clearpage
\section{Question 4} Textbook [Kleinberg \& Tardos] Chapter 4, page 190, problem \#8. \\
 

\clearpage
\section{Question 5} Textbook [Kleinberg \& Tardos] Chapter 4, page 190, problem \#21. \\

\clearpage
\section{Question 6} Textbook [Kleinberg \& Tardos] Chapter 4, page 190, problem \#27. \\

\clearpage
\bibliographystyle{unsrt}
\begin{thebibliography}{9}
	\bibitem{BinomialTheorem} 
	Binomial Theorem,
	\\\texttt{https://en.wikipedia.org/wiki/Binomial\_theorem}
	
	\bibitem{YoungTableau} 
	Young Tableau,
	\\\texttt{https://en.wikipedia.org/wiki/Young\_tableau}
	
\end{thebibliography}

\end{document}